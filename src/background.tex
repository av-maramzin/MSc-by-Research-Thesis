\chapter{Background} \label{backgroud}

\qquad This chapter of the thesis introduces a reader into the context of the work. First, it describes  

\section{Software quality metrics} \label{background-software-quality-metrics}

\subsection{McCabe's cyclomatic complexity (CC)}
\cite{cyclomatic-complexity}

\section{Metrics in the area of parallel computing}

\section{Dependence theory} \label{background-dependence-theory}
\cite{optimizing-compilers-book}

\section{Graph theory} \label{background-graph-theory}
\qquad The work uses some results from the graph theory. In particular, the depth-first search (DFS) graph traversal algorithm and its application to find strongly connected components (SCCs) of graphs. While there are a certain number of variations of these two basic algorithms, the work uses them in the exact form as described in the introduction to algorithms book \cite{introduction-to-algorithms-book}.

\section{Program dependence graph (PDG)}

\cite{pdg}

\subsection{Data dependence graph (DDG)} \label{background-ddg}
\subsection{Memory dependence graph (MDG)} \label{background-mdg}
\subsection{Control dependence graph (CDG)} \label{background-cdg}
\subsection{Program dependence graph (PDG)} \label{background-pdg}

\section{Loop decoupling} \label{background-loop-decoupling}

\cite{iterator-recognition}