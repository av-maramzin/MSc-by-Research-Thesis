\chapter{Software parallelisability metrics tool}
\qquad This chapter describes the tool developed for software source code parallelisability metrics research, how to use it, its software architecture and all the underlying technologies and libraries used during its development. \newline\null\qquad The tool is developed with the C++ language and is almost completely based on the \textsc{LLVM} library of modular and reusable compiler technologies \cite{llvm} \cite{llvm-official-website} and implemented as a set of LLVM passes (see LLVM online documentation for futher technical details \cite{llvm-online-docs}). The tool can be found at \cite{ppar-tool}. All parts of the tool rely heavily on the standard C++ template mechanism and C++ Standard Template Library (STL). \newline \null\qquad The tool operates on the level of LLVM intermediate representation \cite{llvm-online-docs-ir} (LLVM IR) and completely decoupled from input languages as well as from target machine instruction sets. Theoretically the tool can be used for source code parallelisability analysis of any arbitrary programming languages as it does not depend on any exact programming language concepts, data structures and constructs (such as conditional loops, for loops, range-for loops, goto statements, lists, maps, etc). The tool operates on the level of program dependencies (data, control, etc), which are abstracted away from programming languages domain into a separate dependence analysis theory. In order to use a tool, one must provide a way of compiling input language into LLVM intermediate representation. \newline \null\qquad Conceptually the tool does the following. It accepts C/C++ programs as and input. \newline \null\qquad In this project all proposed concepts are being examined with the use of Clang/Clang++ as a front end to transform input C/C++ source code into LLVM instruction set. \newline\null\qquad The remainder of the chapter is structured as follows. Section \ref{implementation-llvm-analyses} briefly describes parts of the LLVM library used in the project. Descriptions are mostly taken from the source code of LLVM and can be studied in more details at \cite{llvm-doxygen-docs}.     

\section{Tool implementation} \label{tool-implementation}
\qquad .There are several LLVM provided analyses being used by the tool.

\subsection{General software architecture} \label{tool-general-software-architecture}
\qquad The tool is implemented withing LLVM pass framework (see \cite{llvm-online-docs-pass-framework}) and architected as a set of LLVM passes, dependent on each other and interacting through the standard mechanism LLVM pass manager provides. There are basically three types of passes in the tool, which are implemented as C++ template classes:
\begin{description}
	
	\item [GraphPass\textbf{\textless NODE,EDGE,PASS\textgreater}] Function analysis pass, which builds dependence graph of a function as well as depencence graphs of all function's loops. This pass stores all the built graphs in the process memory and makes them later accessible for subsequent passes. \textbf{NODE} and \textbf{EDGE} template parameters represent data, associated with each graph node and edge respectively. \textbf{PASS} parameter is used to distinguish different passes, which use the same node and edge types. 

	\item [GraphPrinterPass\textbf{\textless NODE,EDGE,PASS\textgreater}] This pass depends on the \textbf{GraphPass} described above, and dumps its memory content into the files on the hard drive. Dumped files are formatted in accordance with the DOT graph description language and can be visualized with the corresponding tool (such as).    

	\item [DecoupleLoopsPass] Function pass, implemented as a non-template C++ class. Pass runs on a function and computes information for every single function loop. Pass depends on the PDG C++ template specialization of the \textbf{GraphPass} and uses program dependence graphs (PDGs) of function loops to decouple latter into iterator and payload parts. Results are represented as sets of strongly connected components (SCCs). Those SCCs, which belong to the loop payload and those, belonging to the iterator of a loop (there should be only one such SCC). All this information is stored in the process memory and futher accessible for metric computing passes. Detailed algorithms and concepts, underlying the pass implementation, are described in the section \ref{background-loop-decoupling} of the thesis.  

	\item [MetricPass\textbf{\textless METRIC\textgreater}] A C++ template to be specialized and instantiated for every single metric group to be computed. Metrics are computed as function passes, which depend on all passes described above. Different types of metrics, being computed by the tool are described in section \ref{metrics-metric-groups} of the thesis.
	
	\item [MetricCollector] This is a function pass located at the very output end of the whole metric computing pass pipeline. The primary task of that pass is to collect all metrics, computed by \textbf{MetricPass} set of passes, for the given function and report them in the file.     

\end{description} 

\quad These passes rely on some standard LLVM analyses and facilities as well as on the functionality developed withing the current project. Standard LLVM passes, used by the tool are described in section \ref{tool-standard-llvm-analyses} below. Representation of dependence graphs in the memory is described in the section \ref{tool-graph-representation} of this chapter. Section \ref{tool-graph-visualization} describes graph visualisation facilities, provided by the tool. Exact specializations of pass templates, described above, correspond to program dependence graph theory given in section \ref{background-program-dependence-graph}. LLVM details of these specializations are described in section \ref{tool-template-specs}.  

\subsection{Standard LLVM analyses} \label{tool-standard-llvm-analyses}
\qquad The tool uses a number of standard LLVM analyses.

\begin{description}
	
	\item [LoopInfo] This analysis function pass identifies all natural loops withing the given function and assigns a loop depth to every function's basic block. This analysis calculates the nesting structure of loops in the function. For each natural loop identified, this analysis identifies natural loops, contained entirely within the loop and basic blocks that make up the loop.   
	
	\item [DependenceAnalysis] 
	
	\item [PostDominatorTree]
	
\end{description} 

\subsection{Graph representation} \label{tool-graph-representation}
\qquad Since LLVM, as of version 6.0, does not currently provide a standard dependence graph (DG) implementation, custom graph building facilities were implemented in the project as a \textbf{Graph\textless \textsc{NODE},\textsc{EDGE}\textgreater} C++ template. Template expects two parameters, which must be pointers to the \textbf{NODE} and \textbf{EDGE} classes. These classes represent information assosiated with every graph's node and edge correspondingly. The tool uses several types of dependence graphs in its work and these parameters usually end up to be one of the following. NODE parameter is useually either llvm::Instruction or llvm::BasicBlock   

\subsection{Graph visualization facilities} \label{tool-graph-visualization}
\qquad While the main output of the tool is a set of software parallelisability metrics, the tool also accepts a number of side command line options that are useful for debugging to produce additional information, which can supplement bare metric values with some additional insights. Since the tool is based on a set of dependence graphs of programs, it is particularly useful to visualize these graphs.  

\subsection{Template specializations} \label{tool-template-specs}

\section{Tool use} \label{tool-use}

\subsection{Intel(R) Parallel Studio XE 2018}
\qquad Whithin the current project boundaries the tool is used in conjunction with Intel(R) Parallel Studio XE 2018 \cite{intel-parallel-studio}. Intel Parallel Studio XE is a software development product developed by Intel. Parallel Studio is composed of several component parts, each of which is a collection of capabilities. 

These tools help developers boost application performance through superior optimizations and Single Instruction Multiple Data (SIMD) vectorization, integration with Intel® Performance Libraries, and by leveraging the latest OpenMP* 5.0 parallel programming models.

Enhanced optimization reports and integration with Intel® VTune™ Amplifier and Intel® Advisor give developers control over code profiles.

For better performance, it is optimized to take advantage of advanced processor features like multiple cores and wider vector registers, including Intel® Advanced Vector Extensions 512 (Intel® AVX-512) instructions. 

Intel® C++ Compiler in Intel® Parallel Studio XE
