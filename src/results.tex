\chapter{Results}
\label{results}
\qquad The overall picture in the matter of software source code (loops in particular) metrics for  parallelisability resembeles that of the software quality metrics. Software quality (say, maintenance) is a complex notion. To judge about good or bad software design one must posses vast software engeeniring expertise and skills. Metrics like cyclomatic complexity can be used as supplementary to manual analysis, but the values they give must be examined by a human with a deep understanding of the software quality question. Although, their values might sometimes correlate with the understanding of a sound software design, these metrics should not be applied blindly. \newline
\null\qquad Software parallellizability property is not a simplier one. Despite the fact, that all examined metrics have grounds to be proposed and are not randomly selected, they exibit only minor correlations with general parallelisability property and there are always special cases, which break generally established rules and patterns. \newline
\null\qquad However, the working framework, developed withing that  MSc by research project, is ready and can be used for further metrics research and analysis. It is quite easy to add new metrics to the tool. Tool provides visualization facilities for dependence graphs and loop iterator/payload decomposition. New metrics might be added. Alternatively, existing metrics might be fine-tuned as well. This work might be the one on the relatively new direction. Application of machine learning in compilers. Since all machine learning methods require some quantitative features, these loop metrics might be an attempt in their establishment. Modern compilers apply a set of optimizations: loop unrolling, peeling, splitting. Correlations between these metrics and these properties (like loop unrollability) might be examined towards development of machine-learning driven compiler optimizers.\newline
\null\qquad A lot of time has been spent on the development of the tool itself and the first, suitable for analysis, results appeared quite late in the timeframe of the project.         