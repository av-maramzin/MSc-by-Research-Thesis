%%%%%%%%%%%%%%%%%%%%%%%%
%% Sample use of the infthesis class to prepare a thesis. This can be used as 
%% a template to produce your own thesis.
%%
%% The title, abstract and so on are taken from Martin Reddy's csthesis class
%% documentation.
%%
%% MEF, October 2002
%%%%%%%%%%%%%%%%%%%%%%%%

%%%%
%% Load the class. Put any options that you want here (see the documentation
%% for the list of options). The following are samples for each type of
%% thesis:
%%
%% Note: you can also specify any of the following options:
%%  logo: put a University of Edinburgh logo onto the title page
%%  frontabs: put the abstract onto the title page
%%  deptreport: produce a title page that fits into a Computer Science
%%      departmental cover [not sure if this actually works]
%%  singlespacing, fullspacing, doublespacing: choose line spacing
%%  oneside, twoside: specify a one-sided or two-sided thesis
%%  10pt, 11pt, 12pt: choose a font size
%%  centrechapter, leftchapter, rightchapter: alignment of chapter headings
%%  sansheadings, normalheadings: headings and captions in sans-serif
%%      (default) or in the same font as the rest of the thesis
%%  [no]listsintoc: put list of figures/tables in table of contents (default:
%%      not)
%%  romanprepages, plainprepages: number the preliminary pages with Roman
%%      numerals (default) or consecutively with the rest of the thesis
%%  parskip: don't indent paragraphs, put a blank line between instead
%%  abbrevs: define a list of useful abbreviations (see documentation)
%%  draft: produce a single-spaced, double-sided thesis with narrow margins
%%
%% For a PhD thesis -- you must also specify a research institute:
%\documentclass[phd,ilcc,twoside]{infthesis}
\documentclass[mscres,icsa,logo,twoside]{infthesis}

%% For an MPhil thesis -- also needs an institute
% \documentclass[mphil,ianc]{infthesis}

%% MSc by Research, which also needs an institute
% \documentclass[mscres,irr]{infthesis}

%% Taught MSc -- specify a particular degree instead. If none is specified,
%% "MSc in Informatics" is used.
% \documentclass[msc,cogsci]{infthesis}
% \documentclass[msc]{infthesis}  % for the MSc in Informatics

%% Master of Informatics (5 year degree)
% \documentclass[minf]{infthesis}

%% Undergraduate project -- specify the degree course and project type
%% separately
% \documentclass[bsc]{infthesis}
% \course{Artificial Intelligence and Psychology}
% \project{Fourth Year Project Report}

%% Put any \usepackage commands you want to use right here; the following is 
%% an example:
%%\usepackage{natbib}
\usepackage{cite}

\usepackage{float}

\usepackage{graphics}

\usepackage{graphicx}

\graphicspath{{./figs/}}

\usepackage{subcaption}

\usepackage{amsthm}
\usepackage{amsmath,amssymb}
%\usepackage{mathrsfs}
\usepackage{mathtools}

\usepackage{url}
\usepackage{hyperref}

\usepackage{enumitem}
\usepackage{rotating}
\usepackage{pdflscape}

\usepackage{listings}
\usepackage{color}

\lstset{
	backgroundcolor=\color{white}, % choose the background color; you must add \usepackage{color} or \usepackage{xcolor}; should come as last argument
	basicstyle=\normalsize\ttfamily, % the size of the fonts that are used for the code
	breakatwhitespace=false, % sets if automatic breaks should only happen at whitespace
	breaklines=false,                 % sets automatic line breaking
	captionpos=b,                    % sets the caption-position to bottom  
	commentstyle=\itshape\color{purple!40!black}, % comment style
	deletekeywords={...},            % if you want to delete keywords from the given language
	escapeinside={\%*}{*)},          % if you want to add LaTeX within your code
	extendedchars=true,              % lets you use non-ASCII characters; for 8-bits encodings only, does not work with UTF-8
	frame=single,	                   % adds a frame around the code
	keepspaces=true,                 % keeps spaces in text, useful for keeping indentation of code (possibly needs columns=flexible)
	keywordstyle=\bfseries\color{blue},       % keyword style
	language=C,                 % the language of the code
	morekeywords={*,...},            % if you want to add more keywords to the set
	numbers=none,                    % where to put the line-numbers; possible values are (none, left, right)
	numbersep=5pt,                   % how far the line-numbers are from the code
	numberstyle=\tiny\color{black}, % the style that is used for the line-numbers
	rulecolor=\color{black},         % if not set, the frame-color may be changed on line-breaks within not-black text (e.g. comments (green here))
	showspaces=false,                % show spaces everywhere adding particular underscores; it overrides 'showstringspaces'
	showstringspaces=false,          % underline spaces within strings only
	showtabs=false,                  % show tabs within strings adding particular underscores
	stepnumber=2,                    % the step between two line-numbers. If it's 1, each line will be numbered
	stringstyle=\color{red},     % string literal style
	tabsize=4,	                   % sets default tabsize to 2 spaces
	title=\lstname                   % show the filename of files included with \lstinputlisting; also try caption instead of title
}

\theoremstyle{definition}
\newtheorem{definition}{Definition}[section]

\newcommand{\etal}{et~al.}


%% Information about the title, etc.
\title{Software Metrics for Parallelism}
\author{Aleksandr Maramzin}

%% If the year of submission is not the current year, uncomment this line and 
%% specify it here:
% \submityear{1785}

%% Optionally, specify the graduation month and year:
% \graduationdate{February 1786}

%% Specify the abstract here.
\abstract{%
\qquad Parallelism pervades the modern computing world. Almost all modern computing systems provide parallel computing hardware resources to some degree or another. The major problem in the field is that these available resources are not always efficiently utilized. To take the most out of these parallel resources, applications running on them must be parallel as well. \newline 
\null\qquad Despite the progress in parallel programming languages design and increased availability of parallel programming frameworks, writing efficient parallel software from scratch is still a challenging task, mastered by only a few expert programmers. While these experts combine domain knowledge, algorithmic insight and parallel programming skills, most “average” programmers are often lacking skills in at least one of these areas. In this project we investigate methods for providing programmers with real-time feedback on the quality of their code with respect to parallelization opportunities and scalability to address short-comings before they manifest as suboptimal and hard-to-parallelise code.\newline \null\qquad We draw on the experience of the software engineering community and software metrics originally developed to identify low quality sequential code, typically prone to errors and hard to maintain. The goal of this project is to develop novel software parallelisability metrics, which can be used as quality indicators for parallel code and guide the software development process towards better parallel code.
}

%% Now we start with the actual document.
\begin{document}

%% First, the preliminary pages
\begin{preliminary}

%% This creates the title page
\maketitle

%% Acknowledgements
\begin{acknowledgements}
\qquad I would like to express my sincere gratitude to my supervision team, especially to my first advisor Prof. Bjoern Franke for his continuous guidance throughout my MSc by Research study and work. I would not have finished my project without his advices and directions.\newline
\null\qquad Also, I would like to thank all of my classmates for all the valueable discussions we've had throughout the year. Especially, I would like to thank Chris Vasiladiotis for sharing his expertise and his testing framework, which enabled me to get results as early as possible.\newline
\null\qquad And, of course, I am really grateful to my mother for her constant support and encouragement.\newline

\null\quad \textbf{This work was supported in part by the EPSRC Centre for Doctoral Training in Pervasive Parallelism, funded by the UK Engineering and Physical Sciences Research Council (grant EP/L01503X/1) and the University of Edinburgh.}       
 
\end{acknowledgements}

%% Next we need to have the declaration.
\standarddeclaration

%% Finally, a dedication (this is optional -- uncomment the following line if
%% you want one).
% \dedication{To my mummy.}

%% Create the table of contents
\tableofcontents

%% If you want a list of figures or tables, uncomment the appropriate line(s)
% \listoffigures
% \listoftables

\end{preliminary}

%%%%%%%%
%% Include your chapter files here. See the sample chapter file for the basic
%% format.

\chapter{Introduction}
\label{introduction}
\section{General words on parallel programming}
\qquad Parallelism pervades the modern computing world. In the past parallel computations used to be employed only in high performance scientific systems, but now the situation has changed. Parallel elements present in the design of almost all modern computers from small embedded processors to large-scale supercomputers and computing networks. Unfortunately, these immense parallel computing hardware resources are not always fully utilized during computations due to several problems in the field:

\begin{enumerate}[align=left,leftmargin=*]
\item \textbf{Abundance of legacy applications from previous sequential computing era.} That abundance is one source of problems. Legacy applications are not designed to run on parallel machines and, by default, do not take advantage of all underlying hardware resources. Automatic parallelisation techniques have been developed to transform these sequential applications into parallel ones. However, these techniques cannot efficiently deal with some codes in the spectrum of existent applications. Some simple examples of such codes can be found in sections \ref{background-modern-parallelisability-advisor-tools} and \ref{analysis-manual-analysis} of the thesis. More complex codes like pointer-based applications with irregular data structures, applications with loop carried dependencies and entangled control flow have proven to be even more challenging to automatic parallelisation. Very often such programs hide significant amounts of parallelism behind suboptimal implementation constructs and represent meaningful potential for further improvements.
	
\item \textbf{Difficulty of manual parallel programming.} Hidden potential can be realised by writing parallel programs (applications designed to run on parallel systems) manually. However, the task of manual parallel programming is rather challenging by itself. To create efficient and well-designed parallel software programmer must be aware of application's domain field, must have good algorithmic background as well as solid general programming skills and working knowledge of exact parallel programming framework they are using. Most “average” programmers lack some of the necessary skills out of that set, which hinders the potential of manual parallelisation. Sometimes sloppy program parallelisation can even slow sequential programs down due to parallel synchronisation/communication overhead incurred.

\end{enumerate}

\section{Problems of modern parallelizability assistance tools}
\null\qquad While there are some available tools for programmer assistance in software parallelization (like Intel Parallel Studio \ref{background-modern-parallelisability-advisor-tools}), these tools are quite complex and require acquisition of relevant expertise, before they can be effectively used by a programmer. Intel parallelizing compiler (see section \ref{background-modern-parallelisability-advisor-tools}) provides some parallelizability reports to a user. However, these reports mostly have a binary nature: they give "yes"/"no" answers to a parallelizability question. Moreover, Intel compiler has to be conservative in its optimizations in order to preserve program's functional semantic. Sometimes it is too conservative, which leads to detection of different sorts of dependencies in algorithmically parallel loops. In situations, where a programmer chose suboptimal implementations (linked-list instead of a linear array for example) without legitimate reason, Intel compiler will report "a loop is not a candidate for parallelization". More detailed reports would be useful in such situations. Sections \ref{background-modern-parallelisability-advisor-tools} and \ref{analysis-manual-analysis} give some examples of such cases.\newline
\section{Software source code metrics for parallelism}
\qquad In our project we propose to research the question of software metrics for parallelism. This research idea draws on the existent work in the area of \textit{software quality}, where numerous software metrics have been proposed. The most illustrative example of these is the \textit{cyclomatic complexity} measure of a section of a code. In short, cyclomatic complexity is the number of independent paths through control-flow graph of a section of a code. This metric reflects complexity of a source code, and, hence, such properties as readability, maintainability, testability, etc.\newline 
\null\qquad Section \ref{background-software-metrics-in-cs} gives an overview of existent metrics in the general field of computer science and some major related work. The major problem of these metrics is that they do not always correlate with software quality properties and cannot be used blindly alone to judge about those.\newline 
\null\qquad The major questions of this research work are: "Whether these metrics can be adjusted to judge about software parallelizability property?", "Whether there are any metrics, devised for tackling software parallelizability problem?" and "What metrics would be the best for judging about software source code parallelizability?".\newline
\null\qquad As it became clear after related work overview, there hadn't been proposed any software metrics for parallelism. Available metrics for parallelism represent different variations of parallel program \textit{speedup} relative to its sequential version. Section \ref{background-metrics-parallel-computing} presents an overview of related work on that topic.\newline

\section{Software metrics for parallelism as a simple quantitative feedback to a programmer}
\null\qquad The goal of this project is to research the possibility of provision of a more detailed parallelization feedback to a programmer. Instead of binary "yes"/"no" answers, ideally, we would like to see a feedback like: this code is "10\% parallelizible", "80\% parallelizible" or "100\% parallelizible". Where 100\% would mean that this code can be parallelized straight away. Code, which has been assessed as only 10\% is  suboptimal and requires a great deal of work to make it parallel, if it is possible at all. While 80\% parallelizible code requires a small change to turn it into 100\% parallel implementation.\newline
\null\qquad Let's consider two code snippets shown in listings \ref{lst:motivation-0} and \ref{lst:motivation-1}. Both of these code fragments implement the same simple and parallelizible algorithm. While implementation \ref{lst:motivation-0} uses linear array, implementation \ref{lst:motivation-1} uses suboptimal data structure (for that particular task) - linked-list. Intel compiler will give two answers: "yes" for \ref{lst:motivation-0} and "this loop is not a parallel candidate" for \ref{lst:motivation-1}. Here we would like our metrics to provide a programmer with a hint: "\textit{Code in \ref{lst:motivation-1} is not parallelizible, but the algorithm is, and if you rewrite the code, you can run it in parallel}".   

\begin{lstlisting}[float,floatplacement=H,caption={Parallelizible algorithm is implemented with a parallelizible code - parallelizibility metric would report 100\% parallelizible.}, captionpos=b, label=lst:motivation-0]
for (i = 0; i < n; i++) {
	a[i] = a[i] + 1;
}
\end{lstlisting}

\begin{lstlisting}[float,floatplacement=H,caption={Parallelizible algorithm ends up hidden behind non-parallelizible construct (pointer chasing code) - metric would give, say, $\approx$ 20\% parallelizible - there is a great deal of work to do, before it can be turned into a parallel code, but it is possible.}, captionpos=b, label=lst:motivation-1]
while (ptr != nullptr) {
	ptr->value++;
	ptr = ptr->next;
}
\end{lstlisting}      

\section{The structure and the content of the thesis}
\qquad Chapter \ref{backgroud} presents an overview of related work, specifically it describes different software metrics, proposed in the general field of computer science \ref{background-software-metrics-in-cs}, as well as in the subfield of parallel programming \ref{background-metrics-parallel-computing}. It also introduces a reader into the context of the work by giving a necessary background. The major works this thesis is based on are Program Dependence Graph (PDG) intermediate representation proposal \cite{pdg-paper} (described in section \ref{background-program-dependence-graph}) and loop decoupling and iterator recognition work \cite{iterator-recognition-paper} (described in section \ref{background-loop-decoupling}). Metrics, devised and proposed in this work rely heavily on dependence analysis theory \cite{optimizing-compilers-book} (described in section \ref{background-dependence-theory}) and are of dependence-based nature.\newline
\null\qquad Chapter \ref{metrics} introduces actual metrics devised and proposed in this work. It motivates their proposal with some intuitive considerations, gives their definitions and describes how one would use them.\newline
\null\qquad Chapter \ref{ppar-tool} describes PPar (Pervasive Parallelism) tool, developed withing that project. The tool is based on the LLVM compiler components library \cite{llvm-paper}, \cite{llvm-official-website} and computes all proposed metrics on Program Dependence Graphs (PDGs) of program loops. Along with metrics computation it provides graph visualization facilities (see subsection \ref{ppar-tool-graph-visualizations}), which can be used for tool debugging and metrics analysis.\newline
\null\qquad Chapter \ref{benchmarks} describes NAS parallel benchmarks used withing that project for parallelizability metrics research.\newline
\null\qquad Chapter \ref{analysis} presents the evalueation of metrics. It shows different plots of single metric values against software parallelizability property (see section \ref{analysis-data-interpretation-and-visualization}), which allow to see all metrics, correlating with software parallelizability. The chapter also contains clustering analysis (see section \ref{analysis-data-clustering-analysis}), which considers all metrics altogether, as well as decision trees, which represent parallelizability classification derivation based on metric values (see subsection \ref{analysis-decision-tree}). Visualizations of metric values are supplemented with some manual insights into the source code of benchmarks (see subsection \ref{analysis-manual-analysis}). Chapter concludes with application of different machine learning (ML) techniques in order to see which metrics (ML features in that context) are the best for parallelizability ML model training (see section \ref{analysis-statistical-analysis}). 
\chapter{Background} \label{backgroud}
\qquad This chapter of the thesis introduces a reader into the context of the work. First, sections \ref{background-software-metrics-in-cs} and \ref{background-metrics-parallel-computing} give a broad overview of various software source code metrics proposed in the general field of computer science and specifically in the subfield of parallel computing. None of these metrics can be directly used to judge about software source code parallelizability. For tackling that problem there is a need in development of new source code parallelizability metrics.\newline
\null\qquad The field of parallel computing is a really old field, dating back to 1960s and there has been done a huge amount of work. Talking on the very high abstract level, the parallelization of a sequantial program is ultimately constrained by different sorts of dependencies, which exist between program instructions. By that time, there has been established a theory on program dependence analysis. This theory has been well absorbed into many commercial compilers and is being ubiquitously used everywhere in practice. The book \cite{optimizing-compilers-book} conducts a broad compilation of all major results in the field of parallelizing compilers and program dependence theory. Section \ref{background-modern-parallelisability-advisor-tools} briefly describes modern state-of-the-art parallelization tools with Intel's Parallel Studio \cite{intel-parallel-studio} as an example. Section \ref{background-dependence-theory} takes the main theory results from \cite{optimizing-compilers-book} and describes those, which are used in the current project.\newline
\null\qquad This abundance of previous work makes it clear, that there is only one way along which software source code parallelizability metrics study can be conducted. Proposed metrics must use the results of well-established dependence theory. In particular, the whole MSc project work is based on the intermediate dependence-based program representation, namely Program Dependence Graph (PDG). This program dependence data structure has been proposed by in their work \cite{pdg-paper} and is briefly described in section \ref{background-program-dependence-graph}.\newline
\null\qquad Once PDG of a program source code is built, some metrics can be computed straight away. Regular program consists out of modules, subroutines, blocks of code (such as if-statements, for-loops, etc.). As, according to many observations, loops represent 20\% of code, which is executed 80\% of time and are the most seducing piece of workload to parallelize by compiler, this MSc work computes parallelizability metrics on program loops.\newline
\null\qquad The work on loop iterator recognition  and loop decoupling \cite{iterator-recognition-paper} is another foundation this MSc project is based on. Section \ref{background-loop-decoupling} describes the main results of this paper and introduces notions of loop iterator and payload. Metrics proposed and described in chapter \ref{metrics} almost completely rely onto these concepts. The tool, developed for this MSc project (see chapter \ref{ppar-tool}) implements algorithms described in \cite{iterator-recognition-paper} and also uses some graph theory results like strongly connected components (SCCs) finding algorithm and some results from compilers control flow analysis theory, which are briefly outlined in sections \ref{background-graph-theory} and \ref{background-control-flow-analysis}.

\section{Software metrics in computer science} \label{background-software-metrics-in-cs}
\qquad The idea of software source code metrics is definitely not a new one. Quantitative measurements lie as the essence of all exact sciences and there have been numerous efforts to introduce objective metrics in computer science as well. As of the moment computer science quantitative metrics have found their
application mostly in the fields of software quality assessment, software products complexity and software development as a process. These metrics measure
properties of software products such as source code complexity, modularity,
testability and ultimately maintainability. Combined with properties related to
software development processes and projects, they are capable of delivering some
estimates on the total amount of development efforts and associated monetary costs at the end.\newline 
\null\qquad The body of research in this relatively new field is very vast. There are a lot of publications on different types of metrics as well as on their evaluation criteria, axioms the metrics must conform to, their validation, applicability, etc. There has been some efforts to conduct a survey of the field and present an overview of the most important and widespread software metrics to date ([1],[2],[3] to name a few). Work [2] distinguishes two major eras in the field: before 1991, where the main focus was on metrics based on the complexity of the code; and after 1992, where the main focus was on metrics based on the concepts of Object Oriented (OO) systems (design and implementation). Earlier Fabrizio Riguzzi's work [1] dated as 1996 resembles
[2], but also adds some critical insight. Jitender Kumar Chhabra and Varun Gupta in their paper [3] conduct an overview of dynamic software metrics. The later shows that software metrics have gone further from the field of static analysis and moved on to dynamic properties of the software.

\subsection{Source lines of code (SLOC) / lines of code (LOC)}
\label{background-source-lines-of-code}
\qquad Source lines of code (SLOC) or lines of code (LOC) is one of the most widely used, well-known and probably one of the oldest software source code metrics to date. As its name implies, SLOC is measured by counting the number of source codelines in order to give approximate estimation to software size and the total amount of efforts (man-hours) required for development, maintenance, etc. Usually comparisons involve only the order of magnitude of lines of code in the projects. An apparent disadvantage of SLOC metric is that its magnitude on the piece of software does not necessarily correlate with the functionality provided by that piece. SLOC values differ from one language to another and heavily depend on the source code formatting and stylistic factors. Despite all of its disadvantages, SLOC is widely used in software projects size estimations and generally gives good correlations between its magnitude and programming efforts.

\subsection{McCabe's cyclomatic complexity (CC)}
\label{background-cyclomatic-complexity}
\qquad Another well-known software metric is cyclomatic complexity (CC). The metric was first developed by Thomas J. McCabe in 1976 \cite{cyclomatic-complexity-paper}. The metric is based on the control flow graph (CFG) of the section of the code and basically represents the number of linearly independent paths through that section. Mathematically cyclomatic complexity M of a section of the code is defined as M = E – N + 2P, where E is the number of edges, N is the number of nodes, P is the number of connected components in the section's CFG. For example, the piece of code, which CFG is presented on the Figure 1, has cyclomatic complexity equal to 3. The same value 3 follows form it's mathematical equation M = 8 – 7 + 2 = 3. CC metric has been validated both empirically and theoretically and has a lot of applications.

\subsection{Halstead's complexity measures}
\label{background-halsteads-measures}
\qquad Maurice Halstead introduced his software science in 1977 \cite{halstead-book}. In his work Halstead built an analogy between measurable properties of matter (such as volume, mass and pressure of a gas) and those of a source code. He introduced such notions as program length, program volume and program difficulty based on the number of distinct operands and operators in the program.

\subsection{Software cohesion and coupling}
\label{background-cohesion-and-coupling}
\qquad Concepts of software coupling and cohesion were introduced into computer science by Larry Constantine in the late 1960s, when he was working on the field of structured design. The work \cite{cohesion-coupling-paper}, published in 1974 outlines the main results of Larry Constantine's research. Coupling is the degree of interdependence between software modules, while cohesion refers to the degree to which the elements inside the module belong together. These concepts are usually contrasted to each other and often establish inverse proportionality: high coupling often correlates with low cohesion and vice versa. Low coupling and high cohesion are usually a sign of a well-designed system. That system consists of the relatively independent modules. Changes in one part do not usually affect another parts. Degree of reusability is high and particular
system parts (obsolete, malfunctioning, etc.) can be replaced without affecting the rest of the system.

\subsection{Function points}
\qquad Function point is a “unit of measurement” that is used in order to represent the amount of business functionality present in the piece of software. During functional requirements phase of software development, required functionality is identified. Every function is categorized into one of the following types: output, input, inquiry, internal files and external interfaces. Every function is given some amount of function points, which is based on the experience of the past projects. Function Points were proposed by Allan Albrecht in 1979 [7]. Albrecht observed in his research that Function Points were highly correlated to SLOC (3.1) metric.

\subsection{Object-Oriented software metrics}
\qquad In the work [8] Chidamber and Kemerer define a suite of metrics for object oriented designs. They define software metrics for several software properties like cohesion, coupling and complexity. Some examples are presented below:
- Lack of Cohesion in Methods (LCOM): LCOM = (P > Q) ? P – Q : 0, where P
and Q are the numbers of pairs of class methods that do not use / use common class member variables correspondingly.
- Coupling Between Object Classes (CBO): for a class CBO equals to the number
of other classes to which it is coupled. If methods of a class invoke methods or work with member variables of the other class, then classes are coupled.

\subsection{Security metrics for source code structures}
\qquad Software metrics have found their application in the field of source code security as well. Work [9] gives some examples. Described metrics can be used at different stages of software development. Function points (3.5) can be used at initial stages of functional requirements specification. Software cohesion and coupling concepts (3.4) can be considered during later stages of high-level design specification (particular object-oriented software metrics (3.6)). Cyclomatic complexity (3.2), SLOC (3.1), Halstead's complexity measures (3.3) can be used during final and implementation stages for guiding coding efforts. All these metrics give assessments and predictions related to software quality, maintenance, testability, etc. Despite the possibility of correlations between some of these metrics and application parallelisability, these are not designed to directly judge about it.

\section{Metrics in the area of parallel computing}
\label{background-metrics-parallel-computing}
\qquad As in the whole computer science field, there have been proposals of numerous performance metrics in the area of parallel computing as well. Work \cite{parallel-performance-metrics-paper} gives a critical overview of some of the existent parallel performance metrics. These metrics assess software/architecture combinations and use a program running time as their basis. Subsection \ref{background-metrics-speedup-variants} takes some fragments of the work \cite{parallel-performance-metrics-paper} to give a reader an impression of available parallel performance metrics. References to original authors of these metrics are available in the work \cite{parallel-performance-metrics-paper}.     

\subsection{Speedup variants}
\label{background-metrics-speedup-variants}
\qquad The basic question, which arises with program parallelization is "How much faster are we running application on a parallel computer?". The metrics described in this subsection are motivated by that question. While there is a general agreement that a speedup is the ratio:
\begin{equation}
\frac{serial\; execution\; time}{parallel\; execution\; time}
\label{basic-speedup}
\end{equation}
there is a diversity in definitions of parallel and serial execution times.
\begin{enumerate}[align=left,leftmargin=*]
\item \qquad When we use the notion of \textit{relative speedup}, parallel execution time is the time needed to execute a parallel version of a program on a single processor. The final resulting speedup depends on many factors: the number of processors in the system, the interconnection topology used for processor communication, the input dataset for the program, etc. Hence, the final speedup numbers might differ and we may futher introduce a number of such metrics as \textit{maximum relative speedup}, \textit{minimum relative speedup}, etc. 
\item \qquad When we talk about \textit{real speedup}, the role of the \textit{serial execution time} is performed by the time, needed for the fastest known serial algorithm to solve the problem.   
\item \qquad In yet another speedup definition the serial execution time is measured on the fastest serial computer, executing the best known algorithm. The term \textit{absolute speedup} is used for this measure. 
\item \qquad Let $t_{serial}(n)$ and $t_{parallel}(n)$ be asymptotic complexities of a serial and parallel algorithms used to solve a problem respectively. Then the ratio $\frac{t_{serial}(n)}{t_{parallel}(n)}$ is called an \textit{asymptotic speedup}. Asymptotic speedup assumes unlimited number of available processors and is not a function of the number of processors in a system. As with regular speedup, this speedup can futher be classified into relative, real, etc.
\item \qquad If we introduce different parameters (such as hard drive read/write rate, memory latency L, number of processors P, processor cache sizes S, etc.) and write down a final equation for a speedup, then we get so-called \textit{analytical speedup}. 

\end{enumerate} 
\qquad All these enumerated speedup variants can be futher combined or modified to produce a numerous parallel performance metrics. While these metrics can be used to estimate the final speedup a parallel version of a program is going to have on the certain hardware system, they cannot be used for software source code parallelisability feedback and algorithmic parallelizability analysis. In other words, these metrics are not applicable to the problem being tackled in this MSc project.  

\section{Modern parallelisability advisor tools}
\label{background-modern-parallelisability-advisor-tools}
\subsection{Intel(R) Parallel Studio XE 2018}
\qquad Whithin the current project boundaries the tool is used in conjunction with Intel(R) Parallel Studio XE 2018 \cite{intel-parallel-studio}. Intel Parallel Studio XE is a software development product developed by Intel. Parallel Studio is composed of several component parts, each of which is a collection of capabilities. 

These tools help developers boost application performance through superior optimizations and Single Instruction Multiple Data (SIMD) vectorization, integration with Intel® Performance Libraries, and by leveraging the latest OpenMP* 5.0 parallel programming models.

Enhanced optimization reports and integration with Intel® VTune™ Amplifier and Intel® Advisor give developers control over code profiles.

For better performance, it is optimized to take advantage of advanced processor features like multiple cores and wider vector registers, including Intel® Advanced Vector Extensions 512 (Intel® AVX-512) instructions. 

Intel® C++ Compiler in Intel® Parallel Studio XE

\subsection{Automatic parallelisation with Intel(R) C/C++ compilers (ICC)}
\qquad Parallelizing application for the sake of performance improvement can be a time-consuming and skill-requiring activity. For applications, containing relatively simple loops and targeting x86 platforms this task can be automated with the help of Intel C++ compiler \cite{intel-multithreading-guide}. With automatic parallelization ICC detects loops that can be safely and efficiently parallelized and generates multithreaded code. It relieves the programmer from searching for loops that are good candidates for parallel execution, performing dependence analysis and adding parallel compiler directives manually. \newline \null\qquad When it comes to automatic program parallelisation, Intel C/C++ compilers are apparently limited to certain types of loops. \newline \null\qquad Along with actual parallelization Intel C/C++ compilers provide developers with a comprehensive parallelisation reports. \newline \null\qquad Intel C/C++ compiler is used withing the scope and timeframe of the current MSc project as loop parallelisation expert. It's parallelisability reports are transformed into the following format, shown in figure below. That data is used for later statistical learning analysis as labels and parallelisability classifications for different loops of NAS benchmarks (see chapter \ref{benchmarks}).      
   

\section{Dependence theory} \label{background-dependence-theory}
\qquad Modern optimizing and parallelizing compilers use dependence-based approaches to the analyses and transformations they do. Data dependence has been explored since the early days of compilers, dating back to the 1960s, and by now there exist a vast body or research and theory in the domain. The main results and outlines can be found in the optimizing compilers for modern architectures book \cite{optimizing-compilers-book}. Here the brief descriptions fo notions are provided.  
\subsection{Types of dependencies} \label{background-dependence}
\qquad Generally speaking, a dependence is anything that introduces execution order constraints on statements or instructions of the sequential program. Statement S2 is dependent on statement S1, if statement S1 must be executed before statement S2. Dependencies may be broadly classified into two different categories: data and control dependencies. If statement S2 consumes the data, produced by S1, then this type of dependence is called data dependence. If whether S2 will be executed or not depends on the outcome of computation done in S1, then the statement S2 is control-dependent on statement S1. \newline  
\null\qquad Data dependencies are futher subdivided into four subcategories.

\begin{description}
\item [Read After Write (RAW) dependencies]  
\item [Write After Read (WAR) dependencies]    
\item [Write After Write (WAW) dependencies] 
\item [Read After Read (RAR) dependencies] 
\end{description}

\section{Graph theory} \label{background-graph-theory}
\qquad The work uses some results from the graph theory. In particular, the depth-first search (DFS) graph traversal algorithm and its application to find strongly connected components (SCCs) of graphs. While there are a certain number of variations of these two basic algorithms, the work uses them in the exact form as described in the introduction to algorithms book \cite{introduction-to-algorithms-book}.

\section{Control flow analysis} \label{background-control-flow-analysis}
\qquad Control flow analysis \cite{advanced-compiler-design-book}

\section{Program Dependence Graph (PDG)} \label{background-program-dependence-graph}
\qquad A lot of work has been performed over the years in the area of dependence-based program representations and a lot of different  \newline 
\null\qquad The Program Dependence Graph (PDG) is an intermediate dependence-based program representation that makes explicit both the data and control dependencies for each operation in a program. A control flow graph [l, 31 has
been the usual representation for the control flow relationships of a program; the control conditions on which an operation depends can be derived from such a
graph. An undesirable property of a control flow graph, however, is a fixed
sequencing of operations that need not hold. The program dependence graph
explicitly represents both the essential data relationships, as present in the data dependence graph, and the essential control relationships, without the unnecessary sequencing present in the control flow graph.’ These dependence relationships determine the necessary sequencing between operations, exposing potential parallelism. 




\subsection{Data dependence graph (DDG)} \label{background-ddg}
\subsection{Memory dependence graph (MDG)} \label{background-mdg}
\subsection{Control dependence graph (CDG)} \label{background-cdg}
\subsection{Program dependence graph (PDG)} \label{background-pdg}

\section{Loop iterator recognition and loop decoupling} \label{background-loop-decoupling}
\qquad Logically the code, constituting a loop can be divided into two parts. The first part is an actual workload (payload) to be repeated multiple times. The other part is loop iterator the code to control the repetition of the workload.    

\cite{iterator-recognition-paper}


\chapter{Software Parallelisability Metrics} \label{metrics}
\qquad This chapter defines proposed software source code parallelisability metrics and gives the basic intuition behind them. Proposed metrics inherited dependence-based nature from the work \cite{optimizing-compilers-book}. This book is built on and describes the results gathered through countless years of research and tremendous amount of work done in the field of optimizing compilers and high-performance computer architectures. \newline\null\qquad The chapter is structured in the following way. Section \ref{metrics-foundation-and-perspective} puts the metrics work into the context and gives the general perspective from which one has to look at parallelisability metrics. Section \ref{metrics-metric-groups} introduces the actual metrics, along with the basic motivation for them. Metrics are introduced as a set of conceptual groups. Each group has roughly the same intuition and motivation for all its metrics.  

\section{General foundation and perspective of the work} \label{metrics-foundation-and-perspective}

\subsection{Diversity in modern computer languages}
\label{metrics-diversity-in-modern-computer-languages}
\qquad There are thousands of different languages in the modern field of computer science. Computer programming languages have passed a long way from assembly languages operating at the level of native machine instructions to languages operating with concepts at a much higher abstraction levels. The reason behind such a change in the domain of computer languages is the ease, with which a human programmer can write a software. \newline
\null\qquad Unfortunately, this move to a higher-level languages comes with drawbacks as well. With the gain in programmer's productivity, such change also brings losses in software performance. It becomes increasingly difficult for the compiler to translate abstract languages into the sequence of machine instructions effectively. \newline
\null\qquad If we are to use these easy for human comprehension high-level languages, we must have tools for their efficient transformation into the form, suitable for direct execution on different native machine platforms.

\subsection{The modern role of compilers}
\label{metrics-modern-role-of-compilers}
\quad As was outlined in the previous section \ref{metrics-diversity-in-modern-computer-languages}, novadays compilers perform enabling role for the use of different sorts of modern computer languages. \newline
\null\quad In the modern state of the field, the principal role of compiler is to map high-level algorithms onto different sorts of high-performance architectures. The notion of high-performance architectures is really general and usually represents the combining term for all of the following: parallel cluster, multi-core and multi-processor architectures, vector processors, pipelined superscalar processors and all the possible combinations and co-designs of these. \newline
\null\quad Before this mapping can be done, compilers must perform extensive analyses to determine what parts of program computations depend on one another and what parts can be scheduled for parallel execution on high-performance machines. These analyses are mostly dependence-based by their nature.   

\subsection{The famous 80/20 rule}
\label{metrics-famous-80-20-rule}
\qquad Loops and arrays the most fertile ground for optimizations

\subsection{Dependence-based approach to metrics computation}
\label{metrics-dependence-based-approach}
\qquad Program parallelisation of program statements is basically hindered by the execution-order constraints imposed on those statements, which, in turn, are defined by different sorts of program dependencies, which were described in the section \ref{background-dependence-theory} of the thesis.   

\section{Metrics use}
\label{metrics-use}
\qquad Ideally, metrics should provide a quantitative measure of loop's algorithmic parallelisability (say, this loop is 80\% parallelisible). While existent modern tools (like \ref{background-modern-parallelisability-advisor-tools}) give answers only in binary format: yes, this loop has been parallelised or no, it hasn't been parallelised.

\section{Metric Groups}
\label{metrics-metric-groups}
\qquad The whole set of proposed metrics is divided into several concepual groups. To provide an illustrative description of different metrics, let's consider a loop \ref{lst:metrics-loop-example} given below. This loop is taken form EP NAS benchmark (see \ref{benchmarks}).
\begin{lstlisting}[caption={Example loop, taken from EP NAS benchmark}, captionpos=b, label=lst:metrics-loop-example]
for (i = 0; i < NQ; i++) {
	gc = gc + q[i];
}
\end{lstlisting}

\begin{figure}[htb]
	\centering
	\includegraphics[width=\linewidth]{figs/metrics-example-loop-pdg.pdf}
	\caption{Program dependence graph (PDG) of the loop \ref{lst:metrics-loop-example}, as built and visualized by the PPar tool \ref{ppar-tool}.}
	\label{metrics-loop-example-pdg}
\end{figure}

\null\qquad Figure \ref{metrics-loop-example-pdg} above shows program dependence graph of the loop, given in the example. 


\subsection{Loop Proportion Metrics}
\label{metrics-loop-proportion-metrics}
\qquad The first group of metrics computes proportions of the loop. Like Halsted's software science metrics (see \ref{background-software-metrics-in-cs}), it draws an analogy with physical properties of objects (like size, volume, length, etc). This computation happens after PPar tool decouples a loop into iterator and payload code, as described in \ref{background-loop-decoupling}.  
\subsubsection{Loop Absolute Size}
\label{metrics-loop-absolute-size}
\qquad This metric represents the total amount of LLVM IR instructions in the loop. The intuition behind this metric is pretty staighforward: the bigger the loop, the harder it is to parallelize it. The metric has obvious drawbacks. The size of the loop does not, generally speaking, always correlates with loop parallelizability. However, it might be interesting to see, how loop absolute size correlates with loop parallelisability statistically. From figure \ref{metrics-loop-example-pdg} it is visible that the value of the metiric for the given loop \ref{lst:metrics-loop-example} is 15.
\subsubsection{Loop Payload Fraction}
\label{metrics-loop-payload-fraction}
\qquad This metric is complementary to loop absolute size metric and reflects the proportion in which iterator and payload divide the whole loop. There are different considerations behind this metric. For example, if the payload is too small relative to the size of the loop and does not perform significant amount of computations, then parallelization of this loop might not worth the effort.   

\subsubsection{Loop Proper SCCs number}
\label{metrics-loop-proper-sccs-number}
\qquad Once we decoupled a loop into iterator and payload parts, we can split payload part even further. Both iterator and payload are represented by subgraphs in the PDG of the loop. As was described in \ref{background-loop-decoupling}, iterator instructions form a strongly connected component (SCC), which has no incoming dependencies. Payload consists of a set of SCCs. These payload components have different sizes, starting from just 1 instruction and, principally, do not have any upper size limit. If SCC of PDG belongs to the payload of a loop and consists of more than 1 instruction, we call such SCC a \textbf{\textit{proper SCC}}. Usually, such components represent a true dependency in the body of a loop, preventing a loop from parallelization. Thus, the task of parallelizing compilers is to break the edges of such \textbf{\textit{proper (critical )}} SCCs, and transform these SCCs into smaller ones (possibly just 1 instruction). \newline
\null\qquad The example loop from the listing \ref{lst:metrics-loop-example} contains one such proper (critical) SCC. There is a cross-iteration dependency in the body of this loop. The partial sum is being accumulated in the \textit{gc} variable. We can see 2 edges (corresponding to true and anti dependencies) between variable \textit{gc} load and store instructions in the PDG shown on figure \ref{metrics-loop-example-pdg}. Despite the presense of cross-iteration dependency in the loop, Intel C/C++ compiler is capable of its parallelization with reduction techniques.\newline
\null\qquad Simplier loops, like the one shown on the listing \ref{lst:metrics-loop-example-1}, do not contain any SCCs of size greater than 1 (besides iterator SCC). Once we split such loops into iterator and payload parts, all SCCs in the payload are of 1 instruction size. Figure \ref{metrics-example-loop-1-pdg} provides illustration.
  
\begin{lstlisting}[float,floatplacement=H,caption={Parallelizible loop, with no cross-iteration dependencies. Taken from EP NAS benchmark.}, captionpos=b, label=lst:metrics-loop-example-1]
for (i = 0; i < 2 * NK; i++) {
	x[i] = -1.0e99;
}
\end{lstlisting}

\begin{figure}[htb]
	\centering
	\includegraphics[width=\linewidth]{figs/metrics-example-loop-1-pdg.pdf}
	\caption{Program dependence graph (PDG) of the loop \ref{lst:metrics-loop-example-1}, as built and visualized by the PPar tool \ref{ppar-tool}.}
	\label{metrics-example-loop-1-pdg}
\end{figure}

\subsubsection{Loop Critical Payload Fraction}
\label{metrics-loop-critical-payload-fraction}
\qquad In the light of considerations given in the previous section, the fraction between critical and non-critical payload parts might have some correlation with loop parallelizability.  

\subsection{Loop Dependence Metrics}
\label{metrics-loop-dependence-metrics}

\subsection{Loop Cohesion Metrics}
\label{metrics-loop-cohesion-metrics}
\qquad As cohesion and coupling metrics have been proposed for computer software \ref{background-cohesion-and-coupling}, these properties can also be extended in context of this work. \newline
\null\qquad These properties characterize the degree of inter-dependence between different parts of loops. As was shown in the previous 

The main motivation behind the metrics out of this group is the tighter the parts of a loop are coupled together (in terms of dependencies), the harder it is going to be to split and parallalize the loop.   



\chapter{Software parallelisability metrics tool} \label{ppar-tool}
\qquad This chapter describes the tool developed for software source code parallelisability metrics research, how to use it, its software architecture and all the underlying technologies and libraries used during its development. \newline\null\qquad The tool is developed with the C++ language and is almost completely based on the \textsc{LLVM} library of modular and reusable compiler technologies \cite{llvm-paper} \cite{llvm-official-website}. The tool is implemented as a set of LLVM passes (see LLVM online documentation for further technical details \cite{llvm-online-docs}). The tool can be found at \cite{ppar-tool}. All parts of the tool rely heavily on the standard C++ template mechanism and C++ Standard Template Library (STL). \newline \null\qquad The tool operates on the level of LLVM intermediate representation \cite{llvm-online-docs-ir} (LLVM IR) and completely decoupled from input languages as well as from target machine instruction sets. Theoretically, the tool can be used for source code parallelisability analysis of any arbitrary programming languages as it does not depend on any exact programming language concepts, data structures and constructs (such as conditional loops, for loops, range-for loops, goto statements, lists, maps, etc). The tool operates on the level of program dependencies \ref{background-dependence} (data, control, etc), which are abstracted away from programming languages domain into a separate dependence analysis theory \ref{background-dependence-theory}. In order to use the tool, one must provide a way of compiling input language into LLVM intermediate representation. \newline \null\qquad Conceptually the tool does the following. It accepts C/C++ programs as and input. \newline \null\qquad In this project all proposed concepts are being examined with the use of Clang/Clang++ as a front end to transform input C/C++ source code into LLVM instruction set. \newline\null\qquad The remainder of the chapter is structured as follows. Section \ref{implementation-llvm-analyses} briefly describes parts of the LLVM library used in the project. Descriptions are mostly taken from the source code of LLVM and can be studied in more details at \cite{llvm-doxygen-docs}.     

\section{Tool implementation} \label{ppar-tool-implementation}
\qquad .There are several LLVM provided analyses being used by the tool.

\subsection{General software architecture} \label{ppar-tool-general-software-architecture}
\qquad The tool is implemented withing LLVM pass framework (see \cite{llvm-online-docs-pass-framework}) and architected as a set of LLVM passes, dependent on each other and interacting through the standard mechanism LLVM pass manager provides. There are basically three types of passes in the tool, which are implemented as C++ template classes:
\begin{description}
	
	\item [GraphPass\textbf{\textless NODE,EDGE,PASS\textgreater}] Function analysis pass, which builds dependence graph of a function as well as depencence graphs of all function's loops. This pass stores all the built graphs in the process memory and makes them later accessible for subsequent passes. \textbf{NODE} and \textbf{EDGE} template parameters represent data, associated with each graph's node and edge respectively. \textbf{PASS} parameter is used to distinguish different passes, which use the same node and edge types. 

	\item [GraphPrinterPass\textbf{\textless NODE,EDGE,PASS\textgreater}] This pass depends on the \textbf{GraphPass} described above, and dumps its memory content into the files on the hard drive. Dumped files are formatted in accordance with the DOT graph description language and can be visualized with the corresponding tool (such as \cite{graphviz-official-website}).    

	\item [DecoupleLoopsPass] Function pass, implemented as a non-template C++ class. Pass runs on a function and computes information for every single function loop. Pass depends on the PDG C++ template specialization of the \textbf{GraphPass} and uses program dependence graphs (PDGs) of function loops to decouple latter into iterator and payload parts. Results are represented as sets of strongly connected components (SCCs). Those SCCs, which belong to the loop payload and those, belonging to the iterator of a loop (there should be only one such SCC). All this information is stored in the process memory and futher accessible for metric computing passes. Detailed algorithms and concepts, underlying the pass implementation, are described in the section \ref{background-loop-decoupling} of the thesis.  

	\item [MetricPass\textbf{\textless METRIC\textgreater}] A C++ template to be specialized and instantiated for every single metric group to be computed. Metrics are computed as function passes, which depend on all passes described above. Different types of metrics, being computed by the tool are described in section \ref{metrics-metric-groups} of the thesis.
	
	\item [MetricCollector] This is a function pass located at the very output end of the whole metric computing pass pipeline. The primary task of that pass is to collect all metrics, computed by \textbf{MetricPass} set of passes, for the given function and report them in the file.     

\end{description} 

\quad These passes rely on some standard LLVM analyses and facilities as well as on the functionality developed withing the current project. Standard LLVM passes, used by the tool are described in section \ref{tool-standard-llvm-analyses} below. Representation of dependence graphs in the memory is described in the section \ref{tool-graph-representation} of this chapter. Section \ref{tool-graph-visualization} describes graph visualisation facilities, provided by the tool. Exact specializations of pass templates, described above, correspond to program dependence graph theory given in section \ref{background-program-dependence-graph}. LLVM details of these specializations are described in section \ref{tool-template-specs}.  

\subsection{Standard LLVM analyses} \label{ppar-tool-standard-llvm-analyses}
\qquad The tool uses a number of standard LLVM analyses.

\begin{description}
	
	\item [LoopInfo] This analysis function pass identifies all natural loops withing the given function and assigns a loop depth to every function's basic block. This analysis calculates the nesting structure of loops in the function. For each natural loop identified, this analysis identifies natural loops, contained entirely within the loop and basic blocks that make up the loop.   
	
	\item [DependenceAnalysis] 
	
	\item [PostDominatorTree]
	
\end{description} 

\subsection{Graph representation} \label{ppar-tool-graph-representation}
\qquad Since LLVM, as of version 6.0, does not currently provide a standard dependence graph (DG) implementation, custom graph building facilities were implemented in the project as a \textbf{Graph\textless \textsc{NODE},\textsc{EDGE}\textgreater} C++ template. Template expects two parameters, which must be pointers to the \textbf{NODE} and \textbf{EDGE} classes. These classes represent information assosiated with every graph's node and edge correspondingly. The tool uses several types of dependence graphs in its work and these parameters usually end up to be one of the following. NODE parameter is useually either llvm::Instruction or llvm::BasicBlock   

\subsection{Graph visualization facilities} \label{ppar-tool-graph-visualizations}
\qquad While the main output of the tool is a set of software parallelisability metrics, the tool also accepts a number of side command line options that are useful for debugging to produce additional information, which can supplement bare metric values with some additional insights. Since the tool is based on a set of dependence graphs of programs, it is particularly useful to visualize these graphs.  

\subsection{Template specializations} \label{ppar-tool-template-specs}

\section{The tool workflow} \label{ppar-tool-workflow}
\qquad The workflow of the tool can be conceptually divided into 4 phases, following each other in a pipelined fashion:
\begin{enumerate}[align=left,leftmargin=*]
\item \textbf{LLVM part: C/C++ translation into LLVM IR, dependence analysis and loop identification.} The tool is operating on the level of LLVM intermediate representation (IR) \cite{llvm-online-docs-ir}. Clang/Clang++ front-ends translate input C/C++ source code into this IR form. Then, LLVM performs a series of its standard analyses, required by the tool (see section \ref{tool-standard-llvm-analyses}). LoopInfo identifies all the loops in program functions and provides convenient interface for further queries. LLVM builds def-use chains between LLVM IR-level instructions during IR construction. LLVM's dependence analysis identifies data dependencies between memory references in a function. Post-dominance analysis builds a post-dominator tree.
	
\item \textbf{PPar tool program dependence graph (PDG) building part.} In some sense, this part represents the front-end of PPar tool. The tool uses LLVM use-def chains, linking IR instructions, to build data dependence graph (DDG) of a program being examined. After that it uses LLVM dependence analysis and post-dominance tree to build memory dependence graph (MDG) and control dependence graphs (CDG) respectively. The order of these passes does not matter. In principle, they could be done in parallel. Once all three graphs are built, the tool combines all dependencies present in them into a unified program dependence graph (PDG). Detailed descriptions of these graphs can be found in section \ref{background-program-dependence-graph}. All that functionality is done by the corresponding specializations of \textbf{GraphPass\textless NODE,EDGE,PASS\textgreater} template (see \ref{ppar-tool-general-software-architecture}) for every type of dependence graph.   

\item \textbf{Iterator recognition and loop decoupling.} The tool uses results and algorithms, described in the paper \cite{iterator-recognition-paper} to decouple loops into iterator and payload parts (see section \ref{background-loop-decoupling}). This is done by DecoupleLoopsPass (see \ref{ppar-tool-general-software-architecture}).

\item \textbf{PPar tool back-end.} This is the end of the pipeline. Here PPar tool produces its final results. Depending on the purpose, the tool runs here either a set of passes computing parallelisability metrics, or different graph printers (see \ref{ppar-tool-graph-visualizations}) for visual graph analyses and tool debugging. 
		  
\end{enumerate}

\section{Tool use} \label{ppar-tool-use}

\qquad 





\chapter{Benchmarks} \label{benchmarks}
\qquad NAS Parallel Benchmarks \cite{nas-official-website} have been used withing that project. The NAS Parallel Benchmarks (NPB) are a small set of programs designed to help evaluate the performance of parallel supercomputers. The benchmarks are derived from computational fluid dynamics (CFD) applications and consist of five kernels and three pseudo-applications in the original "pencil-and-paper" specification (NPB 1). The benchmark suite has been extended to include new benchmarks for unstructured adaptive mesh, parallel I/O, multi-zone applications, and computational grids.  Problem sizes in NPB are predefined and indicated as different classes. All benchmark specifications are available on the official benchmarks website \cite{nas-official-website}.\newline
\null\qquad This project uses Seoul National University implementation of NAS benchmarks specifications \cite{snu-nas-website}.
\chapter{Analysis}

\section{K-Means clustering}

\section{SVM-based parallelisability analyzer}
\chapter{Results}
\label{results}
\qquad This chapter summarizes the main results of the undertaken MSc by Research 2018 project and describes its workflow as it happened. In short, the main results of the undertaken MSc by Research project can be described with several points:
\begin{enumerate}[align=left,leftmargin=*]
\item \textbf{Software source code parallelizability metrics search has been conducted. MSc bt Research project has been set up.}\newline
\null\qquad A body of literature has been searched through in an attempt to find any software source code metrics, applicable to the software parallelizability problem. While there are a lot of metrics aimed at judging about software quality (maintainability, readability, etc.), none were proposed to judge about software parallelizability. The only metrics in the subfield of parallel computation represent different variations of program execution time speedup ratio. Short report is given in the sections \ref{background-software-metrics-in-cs} and \ref{background-metrics-parallel-computing} of the background chapter \ref{backgroud}.\newline
\null\qquad It was decided, that Program Dependence Graph (PDG) (proposed in the paper \cite{pdg-paper}) is going to be an intermediate program representation, parallelizability metrics would be computed on. Moreover, parallelizability metrics would use loop decoupling and loop iterator recognition results (proposed in the paper \cite{iterator-recognition-paper}) as a prerequisite for their further computation. The first metric to be computed was decided to be \textit{loop payload fraction} (see \ref{metrics-loop-payload-fraction}). LLVM compiler components library \cite{llvm-official-website},\cite{llvm-paper} was chosen for the MSc project to be based on.	
\item \textbf{Development of LLVM-based metric computing tool.}\newline 
\null\qquad This stage took the most of the time and efforts. As a result PPar tool (described in \ref{ppar-tool} and hosted on the GitHub \cite{ppar-tool}) ($\approx$ 4750 C/C++ lines of code) has been developed. The tool development started straight after the first metric ideas were conceived (end of January 2018).\newline
\null\qquad The PDG intermediate representation and all algorithms, proposed and described in \cite{pdg-paper} and \cite{iterator-recognition-paper} have been implemented from the scratch. This took a significant project start-up overheads. Development of dependence graph intermediate representation on top of LLVM IR along with loop decoupling and iterator recognition algorithms (strongly connected components search) took more than a month of time. During that period LLVM DEBUG() prints served as the only debugging and validation means.\newline
\null\qquad After the first graph visualization facilities (thanks to Graphviz and DOT) had been added the project, the first research work has actually started. Along with further validation, debugging and development of the tool, first metrics started to appear on the small hand-written tests. Metric values have been manually validated against PDG and its SCCs DOT graph visualizations. Visualizations of these PDG and their SCCs served as an inspiration for the proposal of a new metrics.\newline
\null\qquad By June 2018, 17 metrics have been devised and integrated into the developed PPar tool framework. Metric values could be obtained on the small set of hand-written tests. These values have been manually verified with graph visualizations. This work has been reported during the Intermediate Progress Review held on the 7$^{th}$ of June 2018.                    
\item \textbf{Devised metric values have been collected on the NAS benchmark suite.}
	
\item \textbf{Analysis of devised parallelizability metrics.} Once all metric values had been gathered for all NAS benchmark loops and all loops had been classified with Intel C/C++ compiler, the final stage of the project could be started.     
	
\end{enumerate}





\qquad The overall picture in the matter of software source code (loops in particular) metrics for  parallelisability resembeles that of the software quality metrics. Software quality (say, maintenance) is a complex notion. To judge about good or bad software design one must posses vast software engeeniring expertise and skills. Metrics like cyclomatic complexity can be used as supplementary to manual analysis, but the values they give must be examined by a human with a deep understanding of the software quality question. Although, their values might sometimes correlate with the understanding of a sound software design, these metrics should not be applied blindly.\newline
\null\qquad Software parallellizability property is not a simplier one. Despite the fact, that all examined metrics have grounds to be proposed and are not randomly selected, their parallelizability correlations are limited and there are always special cases, which break generally established rules and patterns.\newline


\null\qquad However, the working framework, developed withing that  MSc by research project, is ready and can be used for further metrics research and analysis. It is quite easy to add new metrics to the tool. Tool provides visualization facilities for dependence graphs and loop iterator/payload decomposition. New metrics might be added. Alternatively, existing metrics might be fine-tuned as well. This work might be the one on the relatively new direction. Application of machine learning in compilers. Since all machine learning methods require some quantitative features, these loop metrics might be an attempt in their establishment. Modern compilers apply a set of optimizations: loop unrolling, peeling, splitting. Correlations between these metrics and these properties (like loop unrollability) might be examined towards development of machine-learning driven compiler optimizers.\newline
\null\qquad A lot of time has been spent on the development of the tool itself and the first, suitable for analysis, results appeared quite late in the timeframe of the project.         
\chapter{Future work and current limitations}
\label{future-work}
\qquad The major inherent problem of the done MSc research work, is that under given timeframe it was decided to use Intel C/C++ compiler as a parallelizability expert. While ICC is pretty good in program parallelization, there are still cases, where it does not parallelize loops, which could be parallelized. If there was an ideal parallelizing expert available, then some of the red dots in the presented figures would be green, which could slightly change the results. Seoul National University implementation of NAS benchmarks has two versions of the benchmarks: sequential, and the one with added OpenMP pragmas. Ideally, these pragmas could be used as answers to the parallelizability question. It was just technically easier to use ICC compiler to get the answers.\newline  
\null\qquad While there were some little correlations between proposed metrics in their current form and loop parallelizability property, it is clear, that in order to get ideally expected correlation results (if it is principally possible to achieve it with certain precision), these metrics must be tuned, refined and probably supplemented with additional ones. In the current project state it is going to be a way easier. Working research framework has been developed in the form of PPar tool (see \ref{ppar-tool}) and surrounding scripts. All the work from C/C++ source code at the input of LLVM to the final metric figures of Python machine learning scripts has been done. PPar tool is designed in such a way, that it can be easily extended with new metrics without much efforts. Graph visualization facilities can be used to study and refine currently proposed set of metrics.\newline
\null\qquad It is possible to look at the done work from another more general angle. Computation of different numeric features lies at the basis of any machine learning technique. Some loop features have been computed in this work. These features have been examined (withing certain limits) against loop parallelizability property. The same features can be examined against different loop properties, like applicability of different loop optimizations. This work might lie on the path towards machine learning driven optimizing compilers. Apart from loops, some numeric features can be computed for different objects like program data structures, which can enable their identification by compiler.
%% ... etc ...

%%%%%%%%
%% Any appendices should go here. The appendix files should look just like the
%% chapter files.
\appendix
\chapter{Appendix}
\label{appendix}

\qquad The same dataset has been also projected onto 2D plane, as illustrated in the figure \ref{metrics-pca-13-to-2}.

\begin{figure}[htb]
	\centering
	\includegraphics[width=\linewidth]{figs/metrics-pca-13-to-2.png}
	\caption{Visualisation of loop metrics dataset (13-dimensional metric vectors have been projected onto 2d space thanks to PCA algorithm) - blue dots correspond to metric values on single loops.}
	\label{metrics-pca-13-to-2}
\end{figure}

\begin{figure}[h]
	\centering
	\includegraphics[width=\linewidth]{figs/loop-dependencies-number-1.png}
	\caption{\textit{Critical payload true dependencies number} metric on the left and \textit{payload true dependencies number} metric on the right. Red and green dots represent loops, which have not/have been parallelized by ICC compiler correspondingly.}
	\label{loop-dependencies-number-1}
\end{figure}

\begin{figure}[h]
	\centering
	\includegraphics[width=\linewidth]{figs/loop-dependencies-number-2.png}
	\caption{\textit{Critical payload anti dependencies number} metric on the left and \textit{payload anti dependencies number} metric on the right. Red and green dots represent loops, which have not/have been parallelized by ICC compiler correspondingly.}
	\label{loop-dependencies-number-2}
\end{figure}
%% ... etc...

%% Choose your favourite bibliography style here.
%%\bibliographystyle{plain}
\bibliographystyle{unsrt}

%% If you want the bibliography single-spaced (which is allowed), uncomment
%% the next line.
% \singlespace

%% Specify the bibliography file. Default is thesis.bib.
\bibliography{bibliography}

%% ... that's all, folks!
\end{document}
